\chapter{Related Work}
A report on smart grid data management by analytic researchers at Accenture discusses five different data classes for smart grids in \cite{analytics2010achieving}: 
<Churn-rate> : in its broadest sense, is a measure of the number of individuals or items moving out of a collective group over a specific period of time. It is one of two primary factors that determine the steady-state level of customers a business will support.
\begin{enumerate}
 \item \textbf{Operational data}: Represents the electrical behavior of the grid. It includes data such as voltage and current phasors, real and reactive power flows, demand response capacity, distributed energy capacity and power flows, and forecasts for any of these data items.
\item \textbf{Non-operational data}: Represents the condition, health and behavior of assets. It includes master data, data on power quality and reliability, asset stressors, utilization, and telemetry from instruments not directly associated with grid power delivery.
\item \textbf{Meter usage data}: Includes data on total power usage and demand values such as average, peak and time of day. It does not include data items such as voltages, power flows, power factor or power quality data, which are sourced at meters but fall into other data classes.
\item \textbf{Event message data}: Consists of asynchronous event messages from smart grid devices. It includes meter voltage loss/restoration messages, fault detection event messages and event outputs from various technical analytics. As this data is triggered by events, it tends to come in big bursts.
\item \textbf{Metadata}: Is the overarching data needed to organize and interpret all the other data classes. It includes data on grid connectivity, network addresses, point lists, calibration constants, normalizing factors, element naming and network parameters and protocols. Given this scope, managing metadata for a smart grid is a highly challenging task. 
\end{enumerate}
Outlier data have an important influence on accuracy, completeness and self-consistency of data quality. At the same time, it has important events information of electric power grid, such as power rationing, equipment failures and so on \cite{zhang2005new}. Thus, it is of great significance to identify, analyze and deal with outlier data. In residential or industrial electricity, it will produce large amounts of consumption data. Most consumption data are normal. However, there are also some outlier data. The generation of outlier data usually has the following reasons.
\begin{enumerate}
 \item
When electric power system is in operation, measurement and transmission processes of data acquisition system may generate outlier data. For example, outlier changes of data in the transmission are caused by the failure of data transmission system \cite{xu2014application}. Thus, data may lose or confuse in the transmission process.
 \item
The data acquisition system is normal. But outlier changes of electric load are caused by special events, such as: line outage maintenance \cite{dasu1999hunting}. The occurrence of special events usually makes the electricity consumption data in a certain period of time as a null value. This is a kind of outlier phenomenon.
 \item
Significant electricity reforms make electricity consumption habits changed, such as: Ladder-type price \cite{yu2011detecting}. Ladder-type price sets a base to residents, and more than the base price will increase the electricity price. This will lead to changes of electricity habits and result in outlier electricity consumption.
 \item
These conditions may also produce outlier data, such as: records errors, artificial forgery and so on. User's electricity consumption data may hide the user's electricity behavior habits. For data mining of electricity consumption data, on the one hand, it can understand the users’ demands for personalized service; on the other hand, it can detect the users’ stealing behavior and protect the benefits of enterprises.
\end{enumerate}
\frenchspacing 
Applying data mining techniques for power consumption data is a known approach for identifying abnormal usage behavior. Agarwal et al. examined 6 months of data from the UCSD campus, including aggregate power consumption of four buildings in \cite{agarwal2009energy}. Agarwal et al. focuses more on the setup of power meters and provide only simple visualization methods like line charts. Catterson et al. used an approach to monitor old power transformers in \cite{catterson2010online}. Their goal is to proactively search for abnormal behavior that may indicate the transformer is about to fail. Similarly, McArthur et al. searched for anomalies to detect problems with power generation equipment in \cite{mcarthur2005agent}. Jakkula and Cook compared several outlier detection methods to find which is better at identifying abnormal power consumption in \cite{jakkula2010outlier}. J Seem used outlier detection to determine if the energy consumption for a day is significantly different from previous days’ energy consumption in \cite{seem2007using}. This is a known approach for identifying abnormal system behavior.

\frenchspacing There has been many work which already concentrates on securing the smart grid against the cyber- security threats. Rajasegarar et al. discusses anomaly detection techniques in wireless sensor networks considering the limited resources and their effects in \cite{rajasegarar2008anomaly}. Zhang et al. proposes a distribution intrusion detection system which developed and deployed intelligent and multiple analyzing modules in different layers of the smart grid in \cite{zhang2011distributed}. These systems uses support vector machines and Artificial immune systems to detect and classify possible cyber attacks. Fadlullah et al. and Gellings et al. have made steps towards machine to machine communications in [\cite{fadlullah2011early},\cite{gellings2009smart}]. Fadlullah et al. discusses early warning detection technologies in smart grid communication using Gaussian process. With this warning system in place the smart grid control center can forecast the malicious activities to the smart grid which helps it to take measures against such activities. Perl et al. also proposes a solution to outlier detection using Gaussian Mixture model in \cite{perl2009outlier}. 

Ozay et al. uses the machine learning algorithms to detect and classify the measurements as secure and non secure in \cite{ozay2015machine}. It includes well known batch and online learning algorithms (supervised and semi-supervised). It takes advantage of the relationship between the statistical and geographical properties of the attack vectors. These properties are used to detect the unobservable patterns in the attacks. Janetzko et al. describes an analytical and visual approach in detecting anomalies and examining energy consumption data in \cite{janetzko2014anomaly}. The goal of the research was to enable the analyst to understand the power consumption behavior and to be aware of unexpected power consumption values. 

For anomaly detection, machine learning techniques are often applied for detecting security threats, such as data injections or intrusions, instead of detecting bad data measurements. The latter is usually addressed by developing PMU placement algorithms that allow measurement redundancy [\cite{chen2006placement},\cite{tarali2012bad}]. Hink et al. evaluated various machine learning methods to differentiate cyber-attacks from disturbances \cite{hink2014machine}. These methods include OneR, Naive Bayes, SVM, JRipper, and Adaboost. Among these approaches, the results show that applying both JRipper and Adaboost over a three-class space (attack, natural disturbance, and no event) is the most effective approach with the highest accuracy. Pan et al. propose a hybrid intrusion system based on common path mining \cite{pan2015developing}. This approach aggregates both synchrophasor measurement data and audit logs. 

In addition, SVMs have also been used in detection data cyberattacks. For instance, Landford et al. proposes a SVM based approach for detecting PMU data spoofs by learning correlations of data collected by PMUs which are electrically close to each other \cite{landford2015fast}. Due to the time sensitivity of event and anomaly detection on the Smart Grid, a major challenge of applying machine learning techniques to perform these tasks is the efficiency of these approaches. This is especially important for online detection of events. Two different types of approaches have been used to address this challenge. Dimensionality reduction is a commonly used approach to improve learning efficiency by reducing the complexity of the data while preserving the most variance of the original data \cite{van2009dimensionality}. A significant amount of efforts have been made in this direction. For instance, Xie et al. propose a dimensionality reduction method based on principal component analysis for online PMU data processing \cite{xie2014dimensionality}. It has been shown that this method can efficiently detect events at an early stage. A similar technique is used to reduce the dimensionality of synchrophasor data by extracting correlations of the data, and presenting it with their principal components in \cite{dahal2012online}. 

Gaussian Mixture Models have been used extensively in fault detection techniques. Fulufhelo et al. have used GMM classifier in conjunction with the Melfrequency Cepstral Coefficients (MFCC) and Kurtosis features for bearing fault diagnosis in \cite{nelwamondo2006faults}. Detection of a defect in bearings is a process of classifying any of the obtained vibration signals into classes that have been defined during the model generation stage. Li et al. detect unusual patterns in flight data using GMM and is done in \cite{li2016anomaly}. They start transforms the digital flight data into a form applicable for cluster analysis, cluster analysis with a Gaussian Mixture Model (GMM), which characterizes typical behaviors of aircraft systems and temporal patterns of flight operations, detects anomalies based on the norm established in the previous step, safety experts review these anomalies to identify any safety risks at stake.

Li et al. uses probability density function as the measure to detect anomalies which is similar to our approach instead we use weighted log probabilities and an efficient threshold to detect anomalies. A probability density profile for every flight is being constructed after calculating the pdf of every data sample. Abnormal samples are detected by identifying samples with probabilities that is lower than a threshold. This threshold value is associated with the sensitivity of the detection system, and can be adjusted based on the distribution of all samples in a dataset.

Laxhammar et al. also uses GMM and adaptive Kernel Density Estimator to detect anomalies in sea traffic \cite{laxhammar2009anomaly}. The models and algorithms proposed for anomaly detection in the maritime domain are more or less data driven in the sense that normalcy is determined by machine learning algorithms analyzing a relative large set of historical data assumed to reflect normalcy. Anomaly detection results for both models were quite disappointing as the expected trajectory distance observed before detection was considered rather large given the rather constrained normal trajectories and the sweeping and stochastic character of the anomalous trajectories.
\label{sec:related}